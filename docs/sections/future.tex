\begin{itemize}
    \item Develop a PCB for the electronics to mitigate potential issues and eliminate current problems. Currently, connecting IDC 2 with the power off causes the motor drivers to draw power from the Arduino, posing a risk of MCU and computer damage. The root cause of this issue remains elusive.
    
    \item When establishing serial communication with Arduino, the LED linked to pin \texttt{D13} blinks twice, necessitating the power of the motors to be off or another PWM pin to be utilized when connecting the module.
    
    \item Implement joystick functionality for direct control. This feature was intended to facilitate simultaneous adjustment of all actuators, aiding in rod insertion and removal.
    
    \item Complete the implementation of missing command functionalities in the control language.
    
    \item Enhance the design of actuators to streamline assembly. The current placement of screws securing actuators to the base is inaccessible and cumbersome.
    
    \item Address challenges with spring installation and adjustability. Utilize mathematical models or adopt extruder-like systems to regulate spring tension effectively.
    
    \item Refine model tolerances to improve alignment of the rod with bearings and drive pulleys. Explore alternative materials or manufacturing methods to rectify identified defects in the actuator design for 3D printing.
    
    \item Enhance portability by transitioning the system from mains power to battery operation.
    
    \item Develop a static or kinetic model of the robot using analytical methods or machine learning algorithms, leveraging the platform and robot for practical applications.
\end{itemize}