
%% Aduino.h
\subsection{Arduino Main Entry Point}
\begin{minted}[
frame=lines,
framesep=2mm,
baselinestretch=0.8,
fontsize=\footnotesize
]{arduino}
// AlamWare.ino

#include "Memory.h"
#include "Commands.h"

// SerialCommands Setup
char serialCommandBuffer[128];
SerialCommands serialCommands(&Serial, serialCommandBuffer, sizeof(serialCommandBuffer));

void addSerialCommands() {
    serialCommands.SetDefaultHandler(cmdUnrecognized);

    serialCommands.AddCommand(&GHelp);
    serialCommands.AddCommand(&G1);
    serialCommands.AddCommand(&G4);
    serialCommands.AddCommand(&G92);
    serialCommands.AddCommand(&G92_1);
    serialCommands.AddCommand(&G28);
    serialCommands.AddCommand(&G28_1);
    serialCommands.AddCommand(&G90);
    serialCommands.AddCommand(&G91);

    serialCommands.AddCommand(&MHelp);
    serialCommands.AddCommand(&M00);
    serialCommands.AddCommand(&M85);
    serialCommands.AddCommand(&M114);
    serialCommands.AddCommand(&M303);
    serialCommands.AddCommand(&M301);
    serialCommands.AddCommand(&M300);
}

void setup() {
    Serial.begin(250000);
    Serial.print("this is alam");
    addSerialCommands();
}

void loop() {
    if (testing) {
    updateTest();
    } else {
    serialCommands.ReadSerial();

    updateNamedActuators();
    nextCommand();
    }
}
\end{minted}

% Memory.h
\subsection{Persistence Memory Management Class Header}
\begin{minted}[
    frame=lines,
    framesep=2mm,
    baselinestretch=0.8,
    fontsize=\footnotesize
    ]{arduino}
// Memory.h

#ifndef MEMORY_H
#define MEMORY_H

#include <EEPROM.h>

class Memory {
 private:
  float kp;
  float kd;
  float ki;

  uint8_t kpAddress = 10;
  uint8_t kiAddress = 12;
  uint8_t kdAddress = 14;

  int lastPositions[6];
  uint8_t pAddresses[6];

 public:
  Memory();
  ~Memory();

  int GetLastPosition(uint8_t actuatorId);
  void SavePosition(uint8_t actuatorId, int position);
  void GetTunings(float* kp, float* ki, float* kd);
  void SaveTunings(float kp, float ki, float kd);
};

#endif
\end{minted}

% Memory.cpp
\subsection{Persistence Memory Management Class}
\begin{minted}[
    frame=lines,
    framesep=2mm,
    baselinestretch=0.8,
    fontsize=\footnotesize
    ]{arduino}
// Memory.cpp
#include "Memory.h"

Memory::Memory() {
  for (int i = 0; i < 6; i++) {
    pAddresses[i] = 20 + sizeof(float) * i;
  }
}

Memory::~Memory() {
}

void Memory::SavePosition(uint8_t actuatorId, int position) {
  EEPROM.put(pAddresses[actuatorId], position);
}

int Memory::GetLastPosition(uint8_t actuatorId) {
  return lastPositions[actuatorId];
}

void Memory::SaveTunings(float kp, float ki, float kd) {
  EEPROM.put(kpAddress, kp);
  EEPROM.put(kiAddress, ki);
  EEPROM.put(kdAddress, kd);
}

void Memory::GetTunings(float* kp, float* ki, float* kd) {
  EEPROM.get(kpAddress, kp);
  EEPROM.get(kiAddress, ki);
  EEPROM.get(kdAddress, kd);
}
\end{minted}

% Commands.h
\subsection{Commands Definitions}
\begin{minted}[
    frame=lines,
    framesep=2mm,
    baselinestretch=0.7,
    fontsize=\scriptsize
    ]{arduino}
// Commands.h

#ifndef COMMANDS_H
#define COMMANDS_H

#include "SerialCommands.h"

#include "NamedActuators.h"
#include "RodActuator.h"

bool sendingCommand = false;
bool testing = false;
unsigned long startTimeTest;

void updateTest();
void nextCommand();

// G-Code Definitions

void cmdUnrecognized(SerialCommands *, const char *);
void cmdGetGCodes(SerialCommands *);
void cmdDelay(SerialCommands *);
void cmdMoveActuator(SerialCommands *);
void cmdSetGlobalMode(SerialCommands *);
void cmdSetRelativeMode(SerialCommands *);
void cmdSetHome(SerialCommands *);
void cmdSetTarget(SerialCommands *);
void cmdMoveToHome(SerialCommands *);
void cmdMoveToTarget(SerialCommands *);

// M-Code Definitions

void cmdGetMCodes(SerialCommands *);
void cmdStop(SerialCommands *);
void cmdSetTimeout(SerialCommands *);
void cmdGetActuatorPositions(SerialCommands *);
void cmdTestActuators(SerialCommands *);
void cmdStartAutoTuning(SerialCommands *);
void cmdSetPIDConstants(SerialCommands *);

SerialCommand GHelp("G", &cmdGetGCodes);
SerialCommand G4("G4", &cmdDelay);
SerialCommand G1("G1", &cmdMoveActuator);
SerialCommand G28("G28", &cmdMoveToHome);
SerialCommand G28_1("G28.1", &cmdMoveToTarget);
SerialCommand G90("G90", &cmdSetGlobalMode);
SerialCommand G91("G91", &cmdSetRelativeMode);
SerialCommand G92("G92", &cmdSetHome);
SerialCommand G92_1("G92.1", &cmdSetTarget);

SerialCommand MHelp("M", &cmdGetMCodes);
SerialCommand M00("M00", &cmdStop);
SerialCommand M85("M85", &cmdSetTimeout);
SerialCommand M114("M114", &cmdGetActuatorPositions);
SerialCommand M300("M300", &cmdTestActuators);
SerialCommand M303("M303", &cmdStartAutoTuning);
SerialCommand M301("M301", &cmdSetPIDConstants);

void endCmd(SerialCommands *sender) {
  sender->GetSerial()->println("!EOC");
}

void nextCommand() {
  if (!sendingCommand)
    return;
  for (NamedActuator actuator : namedActuators) {
    if (!actuator.object->Next()) {
      return;
    }
  }
  Serial.println("!EOC");
  sendingCommand = false;
}

float getvel(int t) {
  float v;
  float m = 255.0 / 500.0;
  if (t < 500 || t > 4000) {
    v = 0;
  } else if (t < 1000) {  // entre 500 y 1000 ms rampa
    v = m * (t - 500);
  } else if (t < 1500) {  // mantener velocidad máxima por 500 ms
    v = 255;
  } else if (t < 2000) {  // disminuir velocidad
    v = -m * (t - 1500) + 255;
  } else if (t < 2500) {  // mantener velocidad máxima en reversa por 500 ms
    v = 0;
  } else if (t < 3000) {
    v = -m * (t - 2500);
  } else if (t < 3500) {
    v = -255;
  } else if (t < 4000) {
    v = m * (t - 3500) - 255;
  }
  return v;
}

void updateTest() {
  unsigned long t = millis() - startTimeTest;
  Serial.print(t);
  float v = getvel(t);
  Serial.print(";");
  Serial.print(v);
  for (NamedActuator actuator : namedActuators) {
    Serial.print(";");
    actuator.object->TestVelocity(v);
    actuator.object->Update();
    Serial.print(actuator.object->GetPosition());
  }
  Serial.println("");
  if (t >= 4500) {
    testing = false;
    for (NamedActuator actuator : namedActuators) {
      actuator.object->Stop();
    }
    Serial.println("!EOC");
  }
}

void cmdUnrecognized(SerialCommands *sender, const char *cmd) {
  sendingCommand = true;

  sender->GetSerial()->print("Unrecognized command [");
  sender->GetSerial()->print(cmd);
  sender->GetSerial()->println("]");
}

void cmdGetGCodes(SerialCommands *sender) {
  sendingCommand = true;

  sender->GetSerial()->println("G: Show this help");
  sender->GetSerial()->println("G1: Move actuator");
  sender->GetSerial()->println("G4: Pause robot by a determined P<time>");
  sender->GetSerial()->println("G28: Move actuators to home position");
  sender->GetSerial()->println("G28.1: Move actuators to target position");
  sender->GetSerial()->println("G90: Set global positioning mode");
  sender->GetSerial()->println("G91: Set relative positioning mode");
  sender->GetSerial()->println("G92: Set home position");
  sender->GetSerial()->println("G92.1: Set target position");
}

void cmdGetMCodes(SerialCommands *sender) {
  sendingCommand = true;

  sender->GetSerial()->println("M: Show this help");
  sender->GetSerial()->println("M00: Stop actuators");
  sender->GetSerial()->println("M85: Set command timeout with P<milliseconds> or S<seconds>");
  sender->GetSerial()->println("M114: Report actual position of actuators");
  sender->GetSerial()->println("M300: Test actuators");
  sender->GetSerial()->println("M301: Set PID constants with P<value> I<value> D<value>");
  sender->GetSerial()->println("M303: Starts autotuning");
}

void cmdDelay(SerialCommands *sender) {
  sendingCommand = true;

  // sender->GetSerial()->println("Starting delay");
  char *arg = sender->Next();
  if (arg != NULL) {
    char parameter = arg[0];
    float value = atof(arg + 1);
    if (parameter == 'P') {
      delay(value);
    }
  }
}

void cmdMoveActuator(SerialCommands *sender) {
  sendingCommand = true;
  uint8_t speed = 255;
  char *arg;
  while ((arg = sender->Next()) != NULL) {
    char id = arg[0];
    float value = atof(arg + 1);
    if (id == 'F') {
      speed = value;
      continue;
    }
    RodActuator *actuator = getActuatorByName(id);
    actuator->Move(value);
    actuator->SetMaxSpeed(speed);
  }
}

void cmdStop(SerialCommands *sender) {
  sendingCommand = true;
  for (NamedActuator actuator : namedActuators) {
    actuator.object->Stop();
  }
}

void cmdSetTimeout(SerialCommands *sender) {
  sendingCommand = true;
  float maximumTimeout = 60e3;
  float timeout = 5000;
  char *arg = sender->Next();
  if (arg != NULL) {
    char parameter = arg[0];
    float value = atof(arg + 1);
    if (parameter == 'P') {
      timeout = value;
    } else if (parameter == 'S') {
      timeout = value * 1000;
    }
    if (timeout > maximumTimeout)
      timeout = maximumTimeout;
    for (NamedActuator actuator : namedActuators) {
      actuator.object->SetTimeout(timeout);
    }
  } else {
    timeout = namedActuators[0].object->GetTimeout();
  }
  sender->GetSerial()->print("Timeout:");
  sender->GetSerial()->println(timeout);
}

void cmdGetActuatorPositions(SerialCommands *sender) {
  sendingCommand = true;

  int decimalPlaces = 5;

  String positions = "";
  for (int i = 0; i < N_ACTUATORS; ++i) {
    String name = (String)namedActuators[i].name;
    float position = namedActuators[i].object->GetPosition();
    positions += name + ":" + String(position);
    if (i < N_ACTUATORS - 1) {
      {
        positions += " ";
      }
    }
  }
  sender->GetSerial()->println(positions);
}

void cmdStartAutoTuning(SerialCommands *sender) {
  sendingCommand = true;

  char *arg;
  while ((arg = sender->Next()) != NULL) {
    char id = arg[0];
    Serial.print("Starting auto tuning in actuator ");
    Serial.println(id);
    RodActuator *actuator = getActuatorByName(id);
    actuator->AutoTune();
  }
}

void cmdTestActuators(SerialCommands *sender) {
  testing = true;
  startTimeTest = millis();
  sender->GetSerial()->print("t");
  sender->GetSerial()->print(";PWM");
  for (NamedActuator actuator : namedActuators) {
    sender->GetSerial()->print(";");
    sender->GetSerial()->print(actuator.name);
  }
  sender->GetSerial()->println("");
}

void cmdSetPIDConstants(SerialCommands *sender) {
  sendingCommand = true;

  float kp = 0;
  float ki = 0;
  float kd = 0;

  char *arg;
  while ((arg = sender->Next()) != NULL) {
    char id = arg[0];
    float value = atof(arg + 1);
    if (id == 'P')
      kp = value;
    else if (id == 'I')
      ki = value;
    else if (id == 'D')
      kd = value;
  }

  for (NamedActuator actuator : namedActuators) {
    actuator.object->SetPIDConstants(kp, ki, kd);
  }

  sender->GetSerial()->print("Kp:");
  sender->GetSerial()->print(kp);
  sender->GetSerial()->print(" Kd:");
  sender->GetSerial()->print(kd);
  sender->GetSerial()->print(" Ki:");
  sender->GetSerial()->println(ki);
}

void cmdSetGlobalMode(SerialCommands *sender) {
  for (NamedActuator actuator : namedActuators) {
    actuator.object->SetMoveMode(actuator.object->Global);
  }
}

void cmdSetRelativeMode(SerialCommands *sender) {
  for (NamedActuator actuator : namedActuators) {
    actuator.object->SetMoveMode(actuator.object->Relative);
  }
}

void cmdSetHome(SerialCommands *sender) {
  sendingCommand = true;
  char *arg;
  while ((arg = sender->Next()) != NULL) {
    char id = arg[0];
    float value = atof(arg + 1);
    RodActuator *actuator = getActuatorByName(id);
    actuator->SetHomePosition(value);
  }
}

void cmdSetTarget(SerialCommands *sender) {
  sendingCommand = true;
  char *arg;
  while ((arg = sender->Next()) != NULL) {
    char id = arg[0];
    float value = atof(arg + 1);
    RodActuator *actuator = getActuatorByName(id);
    actuator->SetTargetPosition(value);
  }
}

void cmdMoveToHome(SerialCommands *sender) {
  sendingCommand = true;
  char *arg;
  arg = sender->Next();
  if (arg == NULL) {
    for (NamedActuator actuator : namedActuators) {
      actuator.object->MoveToHome();
    }
  } else {
    char id = arg[0];
    float value = atof(arg + 1);
    RodActuator *actuator = getActuatorByName(id);
    actuator->MoveToHome();

    while ((arg = sender->Next()) != NULL) {
      id = arg[0];
      float value = atof(arg + 1);
      actuator = getActuatorByName(id);
      actuator->MoveToHome();
    }
  }
}

void cmdMoveToTarget(SerialCommands *sender) {
  sendingCommand = true;
  char *arg;
  arg = sender->Next();
  if (arg == NULL) {
    for (NamedActuator actuator : namedActuators) {
      actuator.object->MoveToTarget();
    }
  } else {
    char id = arg[0];
    float value = atof(arg + 1);
    RodActuator *actuator = getActuatorByName(id);
    actuator->MoveToTarget();

    while ((arg = sender->Next()) != NULL) {
      id = arg[0];
      float value = atof(arg + 1);
      actuator = getActuatorByName(id);
      actuator->MoveToTarget();
    }
  }
}

#endif
\end{minted}

% NamedActuators.h
\subsection{Actuators Instances}
\begin{minted}[
    frame=lines,
    framesep=2mm,
    baselinestretch=0.7,
    fontsize=\scriptsize
    ]{arduino}
// NamedActuators.h
#ifndef NAMEDACTUATORS_H
#define NAMEDACTUATORS_H

#include "RodActuator.h"

// 1.1
#define A_ENA 21
#define A_ENB 24
#define A_IN1 13
#define A_IN2 12
// 1.3
#define B_ENA 20
#define B_ENB 23
#define B_IN1 6
#define B_IN2 7
// 2.2
#define C_ENA 3
#define C_ENB 66
#define C_IN1 10
#define C_IN2 11
// 2.1
#define I_ENA 19
#define I_ENB 65
#define I_IN1 9
#define I_IN2 8
// 2.3
#define J_ENA 2
#define J_ENB 64
#define J_IN1 5
#define J_IN2 4
// 1.2
#define K_ENA 18
#define K_ENB 22
#define K_IN1 44
#define K_IN2 45

enum Actuators {
  A, B, C, I, J, K,
  N_ACTUATORS
};

struct NamedActuator {
  char name;
  RodActuator *object;
};

// Posiciones
volatile long angPosition[N_ACTUATORS] = {0, 0, 0, 0, 0, 0};

//
// Interrupt Handlers
//
static void A_interruptHandler() {
  if (digitalRead(A_ENA) != digitalRead(A_ENB))
    angPosition[A]++;
  else
    angPosition[A]--;
}
static void B_interruptHandler() {
  if (digitalRead(B_ENA) != digitalRead(B_ENB))
    angPosition[B]++;
  else
    angPosition[B]--;
}

static void C_interruptHandler() {
  if (digitalRead(C_ENA) != digitalRead(C_ENB))
    angPosition[C]++;
  else
    angPosition[C]--;
}

static void I_interruptHandler() {
  if (digitalRead(I_ENA) != digitalRead(I_ENB))
    angPosition[I]++;
  else
    angPosition[I]--;
}

static void J_interruptHandler() {
  if (digitalRead(J_ENA) != digitalRead(J_ENB))
    angPosition[J]++;
  else
    angPosition[J]--;
}

static void K_interruptHandler() {
  if (digitalRead(K_ENA) != digitalRead(K_ENB))
    angPosition[K]++;
  else
    angPosition[K]--;
}

NamedActuator namedActuators[N_ACTUATORS] = {
    {'A', new RodActuator(A_ENA, A_ENB, A_IN1, A_IN2, angPosition[A], A_interruptHandler)},
    {'B', new RodActuator(B_ENA, B_ENB, B_IN1, B_IN2, angPosition[B], B_interruptHandler)},
    {'C', new RodActuator(C_ENA, C_ENB, C_IN1, C_IN2, angPosition[C], C_interruptHandler)},
    {'I', new RodActuator(I_ENA, I_ENB, I_IN1, I_IN2, angPosition[I], I_interruptHandler)},
    {'J', new RodActuator(J_ENA, J_ENB, J_IN1, J_IN2, angPosition[J], J_interruptHandler)},
    {'K', new RodActuator(K_ENA, K_ENB, K_IN1, K_IN2, angPosition[K], K_interruptHandler)},
};

RodActuator *getActuatorByName(char name) {
  for (int i = 0; i < N_ACTUATORS; ++i) {
    if (namedActuators[i].name == name) {
      return namedActuators[i].object;
    }
  }
  return nullptr;
}

void updateNamedActuators() {
  for (NamedActuator actuator : namedActuators) {
    actuator.object->Update();
  }
}

#endif
\end{minted}

% Rodactuator.h
\subsection{Actuator Class Header}
\begin{minted}[
frame=lines,
framesep=2mm,
baselinestretch=0.7,
fontsize=\scriptsize
]{arduino}
//RodActuator.h
#ifndef RODACTUATOR_H
#define RODACTUATOR_H

#include <Arduino.h>
#include <QuickPID.h>
#include <sTune.h>

class RodActuator {
 public:  
  RodActuator(uint8_t ENA, uint8_t ENB, uint8_t IN1, uint8_t IN2,
    volatile long &angPosition, void (*interruptHandler)());

  enum MoveMode {
    Global,
    Relative
  };

  void Stop();
  void Update();
  void SetPIDConstants(float Kp, float Ki, float Kd);
  void Move(float desiredPosition);
  float GetPosition();
  int GetTimeout();
  void TestVelocity(float velocity);
  void SetLeftLimit(float leftLimit);
  void SetRightLimit(float rightLimit);
  void SetMaxSpeed(uint8_t maxSpeed);
  void SetMoveMode(MoveMode);
  void SetTimeout(float timeout);
  void MoveToHome();
  void MoveToTarget();
  void SetHomePosition(float newHomePosition);
  void SetTargetPosition(float newSecondHomePosition);
  float GetHomePosition();
  float GetTargetPosition();
  void AutoTune();
  bool Next();

 private:
  float homePosition = 0;
  float targetPosition = 0;

  bool isStable(float threshold);
  volatile long &angPosition;
  uint8_t IN1;  // Pin forward velocity
  uint8_t IN2;  // Pin backwards velocity
  uint8_t ENA;  // Pin encoder A
  uint8_t ENB;  // Pin encoder B

  float ticksRev = 7.0;     // ticks in angPosition that makes a rev
  float gearRatio = 210.0;  // reducion gear of gearRatio:1
  float R = 5.3;
  float positionFactor;

  uint8_t speed = 255;
  unsigned long timeout = 5000;
  unsigned long timer = millis();
  float position = 0;
  float lastPosition = 0;
  float desiredPosition = 0;
  float lastAngvelocity = 0;
  float angVelocity = 0;  // motor angular position

  MoveMode moveMode = Global;

  bool next = true;
  bool move = false;
  float Kp, Kd, Ki;
  float outputMin = -255;
  float outputMax = 255;
  float outputSpan;
  bool startup = false;
  QuickPID PID;  // Controlador PID

  bool test = false;
  bool tune = false;
  sTune tuner = sTune(&position, &angVelocity, tuner.NoOvershoot_PI, tuner.directIP, tuner.printALL);

  float _leftLimit = -2000;
  float _rightLimit = 2000;

  void updateVelocity();
  void updatePosition();
};

#endif
\end{minted}

% Rodactuator.h
\subsection{Actuator Class}
\begin{minted}[
frame=lines,
framesep=2mm,
baselinestretch=0.7,
fontsize=\scriptsize
]{arduino}
//RodActuator.cpp

#include "RodActuator.h"

#include <Arduino.h>

#define DEBUG false

RodActuator::RodActuator(uint8_t ENA, uint8_t ENB, uint8_t IN1, uint8_t IN2,
    volatile long &angPosition, void (*interruptHandler)()) :
    ENA(ENA), ENB(ENB), IN1(IN1), IN2(IN2), angPosition(angPosition) {
  pinMode(ENA, INPUT);
  pinMode(ENB, INPUT);
  pinMode(IN1, OUTPUT);
  pinMode(IN2, OUTPUT);

  attachInterrupt(digitalPinToInterrupt(ENA), interruptHandler, CHANGE);

  // Calculate position factor
  // 1 Revolución en el encoder son 7 cambios
  // La relación de engranages de 235:1
  // El valor de _position está en revoluciones
  // 1/(7*235) = 6.0790273556231e-4
  positionFactor = 2.0 * R * PI / (ticksRev * gearRatio);

  // Inicializar control PID
  PID = QuickPID(&position, &angVelocity, &desiredPosition);
  PID.SetOutputLimits(outputMin, outputMax);
  PID.SetSampleTimeUs(2030);
  PID.SetDerivativeMode(QuickPID::dMode::dOnError);
  PID.SetProportionalMode(QuickPID::pMode::pOnError);
  PID.SetAntiWindupMode(QuickPID::iAwMode::iAwCondition);
  PID.SetMode(QuickPID::Control::automatic);
  PID.SetTunings(22.1, 67, 3.8632);
}

void RodActuator::Update() {
  updatePosition();

  if (tune) {
    switch (tuner.Run()) {
      case tuner.sample:
        updateVelocity();
        break;

      case tuner.tunings:
        angVelocity = 0;
        updateVelocity();
        tuner.GetAutoTunings(&Kp, &Ki, &Kd);
        SetPIDConstants(Kp, Ki, Kd);
        tune = false;
        next = true;
        break;

      default:
        break;
    }
  }

  else if (move) {
    if (startup && position > desiredPosition - 10) {
      startup = false;
      angVelocity = 0.01 * angVelocity;
      updateVelocity();
      PID.SetMode(QuickPID::Control::manual);
      PID.SetMode(QuickPID::Control::automatic);
    }
    PID.Compute();
    updateVelocity();
  }

  else if (test) {
    updateVelocity();
  }

  lastPosition = position;
  lastAngvelocity = angVelocity;

  if (isStable(1.5)) {
    next = true;
  }

  if (!next && move) {
    unsigned long currentTime = millis();
    if (currentTime - timer > timeout) {
      next = true;
    }
  }
}

void RodActuator::Stop() {
  angVelocity = 0;
  next = true;
  move = false;
  test = false;
  updateVelocity();
  updatePosition();
}

void RodActuator::Move(float desiredPosition) {
  next = false;
  timer = millis();
  move = true;
  startup = true;
  if (moveMode == MoveMode::Global) {
    RodActuator::desiredPosition = desiredPosition;
  } else if (moveMode == MoveMode::Relative) {
    RodActuator::desiredPosition += desiredPosition;
  }
}

float RodActuator::GetHomePosition() {
  return homePosition;
}

float RodActuator::GetTargetPosition() {
  return targetPosition;
}

void RodActuator::SetHomePosition(float newHome) {
  RodActuator::homePosition = newHome;
}

void RodActuator::SetTargetPosition(float newSecondHome) {
  RodActuator::targetPosition = newSecondHome;
}

void RodActuator::MoveToHome() {
  float desiredPosition = homePosition;
  if (moveMode == MoveMode::Relative) {
    desiredPosition = homePosition - desiredPosition;
  }
  Move(desiredPosition);
}

void RodActuator::MoveToTarget() {
  float desiredPosition = targetPosition;
  if (moveMode == MoveMode::Relative) {
    desiredPosition = targetPosition - desiredPosition;
  }
  Move(desiredPosition);
}

float RodActuator::GetPosition() {
  return position;
}

void RodActuator::SetPIDConstants(float Kp, float Ki, float Kd) {
  Kp = Kp;
  Ki = Ki;
  Kd = Kd;
  PID.SetTunings(Kp, Ki, Kd);
}

bool RodActuator::Next() {
  return next;
}

void RodActuator::SetMaxSpeed(uint8_t maxSpeed) {
  if (maxSpeed < 50) {
    maxSpeed = 50;
  }
  RodActuator::speed = maxSpeed;
}

void RodActuator::TestVelocity(float velocity) {
  RodActuator::angVelocity = velocity;
  test = true;
}

void RodActuator::SetTimeout(float timeout) {
  RodActuator::timeout = timeout;
}

int RodActuator::GetTimeout() {
  return timeout;
}

void RodActuator::AutoTune() {
  tune = true;
  next = false;
  tuner.Configure(1000, 200, 100, 0, 3, 1, 500);
  tuner.SetEmergencyStop(1000);
}

void RodActuator::SetMoveMode(MoveMode moveMode) {
  RodActuator::moveMode = moveMode;
}

// =========================================
// Private methods
// =========================================
bool RodActuator::isStable(float threshold = 1) {
  return abs(position - desiredPosition) <= threshold;
}

void RodActuator::updateVelocity() {
  uint8_t vel1 = 0;
  uint8_t vel2 = 0;
  float minInput = 0;
  uint8_t minPWM = 0;
  if (angVelocity > minInput) {
    // vel1 = abs((float)_angVelocity / (float)_outputMax) * (float)_speed;
    vel1 = map(abs(angVelocity), minInput, outputMax, minPWM, RodActuator::speed);
    if (vel1 < 40)
      vel1 = 0;
  } else if (angVelocity < -minInput) {
    // vel2 = abs((float)_angVelocity / (float)_outputMax) * (float)_speed;
    vel2 = map(abs(angVelocity), minInput, outputMax, minPWM, RodActuator::speed);
    if (vel2 < 40)
      vel2 = 0;
  }
  // Serial.println(vel1);
  if (DEBUG && (move && !next)) {
    Serial.print("Velocity: ");
    Serial.print(angVelocity);
    Serial.print(" ");
    Serial.print(vel1);
    Serial.print(" ");
    Serial.print(vel2);
    Serial.print("  Position: ");
    Serial.print(position);
    Serial.println();
  }
  analogWrite(IN1, vel1);
  analogWrite(IN2, vel2);
}

void RodActuator::updatePosition() {
  position = angPosition * positionFactor;
}
\end{minted}